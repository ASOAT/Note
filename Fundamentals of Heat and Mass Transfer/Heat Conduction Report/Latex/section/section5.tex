其他影响因素主要考虑房间的窗户和墙壁设置。根据傅里叶定律,墙体对房间温度的影响主要与墙体的厚度以及导热系数有关。一般而言,承重墙、非承重墙以及楼板的厚度都有所不同,会影响隔壁房间对待测房间的温度影响。在考虑开启空调和地暖的情况下,选择导热系数低的墙体材料,例如硬泡聚氨酯复合板、硬泡聚氨酯陶瓷等都有助于增加墙壁的保温隔热性能,减少能量的消耗。

除此以外,墙壁外立面的颜色等也会影响室内温度。以黑色外立面的墙壁为例,在夏天环境温度为32℃时,墙体的温度较一般墙体上升了5℃,对室内的制冷造成了一定的负担。但同时这一类墙体在冬天也能为室内的采暖减轻负担。

在考虑建筑结构上出现承重墙与非承重墙的时候,墙体结构也会对导热产生影响。部分非承重墙为了节省建筑材料会采用中空的结构。由于空气的导热系数远小于墙体材料的导热系数,因此该类非承重墙能够提升室内的保温效果。非承重墙往往用于隔离出卫生间、厨房等区域,这类区域在使用时往往会产生大量的热量,例如在卫生间内淋浴、在厨房内烧菜。因此采用中空结构的非承重墙能够在冬天很好地保持这些区域的温度,提高居住的舒适度。

除了墙体以外,窗户的结构也对室内温度有着显著的影响。首先从结构上而言,目前大范围使用的双层玻璃有着与中空非承重墙一样的传热学原理,能够增强窗户的隔热效果。同时也出现了类似于三层中空玻璃的结构,亦或是向玻璃的中空层中填充氩气等惰性气体,这些设计都提升玻璃的隔热效果,减少室内制冷或采暖时的能量消耗。

同时窗户的面积也对室内温度有着显著的影响,但是由于不同户型的墙面面积不同,单纯比较窗户的面积没有任何参考的价值,因此引入窗墙比的概念。因为玻璃的传热系数大于墙体,因此窗墙比越大的建筑,采暖时的能量损耗也越大。《公共建筑节能设计标准》GB50189-2005第4.2.4条规定:建筑每个朝向的窗(包括透明幕墙)墙面积比均不应大于0.70。当窗(包括透明幕墙)墙面积比小于0.40时,玻璃(或其他透明材料)的可见光透射比不应小于0.40。但是当下人们对于建筑的审美往往要求房间能够有较大的窗墙比,从而获得更好的采光并且能让房间看上去更加通透,更加适合居住。因此设计出隔热性能更好的透明窗户方案抑或是可随户主意愿调节采光度的非透明玻璃幕墙都不失为有前景的研究方向。
